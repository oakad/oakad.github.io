Ультиматум.

\vspace{1ex}\noindent
Волосы дыбом!
\vspace{1ex}

Австрия в начале ультиматума перечисляла все проступки сербского правительства
(отдыхавшего в это время, по большей части, в Австрии). Главными проступками
были \enquote{попустительство террористам и прочим антиавстрийским элементам}.
Белград безосновательно (на тот момент) обвиняли в причастности к сараевским
убийствам.

Далее шли требования:

\begin{itemize}
\item Торжественно и публично осудить сербским правительством всякой агитации
против Австро -- Венгрии.
\item В приказе по сербской армии король Сербии должен был также осудить
антиавстрийские выпады среди сербского генералитета и офицерства.
\item Австрия требовала ужесточить контроль над сербскими СМИ со стороны
сербского правительства для снижения накала антиавстрийской пропаганды, ведущей
к войне.
\item Австрия требовала закрыть враждебные Австрии общественные организации в
Сербии.
\item Был представлен список сербских офицеров и чиновников, замешанных в
антиавстрийской деятельности. Австрия требовала увольнения этих господ с
государственной службы.
\item Австрия требовала строгого наказания всех лиц с сербской стороны,
замешанных в сараевских событиях.
\end{itemize}

Сербия приняла все условия австрийского ультиматума.

Кроме одного (п.5 -- 6): Сербы отказались допустить на свою территорию австрийскую
следственную группу для расследования сараевских событий.

Ещё раз перечитайте условия ультиматума, принятые сербами и перечитайте условие,
сербами яростно отвергнутое. Взвесьте на весах здравого смысла. Обычно ведь
пишут, что сербы восстали против страшного покушения на собственный суверенитет!
Что сербов возмутило то, что Автрия диктует им условия внутренней политики и
кадрового резерва. А на деле, все эти страшные условия по увольнению офицеров,
закрытию газет, приказам по армии, указания королю Сербии, что и как говорить и
пр. сербы приняли. А вот австрийскую полицию копать заговор против Фердинанда
не захотели принять! Обычную следственную группу. Для меня это очень
показательно. На любой позор были согласны, только не копайте у нас, кто отдавал
приказ о терроре. Вильгельм, которому переслали ответ сербского правительства
написал на докладе: \enquote{Блестящий результат за 48 часов! Он превзошёл все
ожидания. Отпадают основания для войны}. И даже попросил Вену использовать
открывшуюся принятыми условиями ультиматума возможность для невоенного решения.

Сербы прекрасно знали своё состояние после своих двух недавних войн, не питали
иллюзий по отношению к мощи российской армии, совершенно ненавидели болгар,
т.е шансы свои оценивали трезво. Но по формальному признаку ультиматум был ими
отвергнут. 25 июля 1914 года австрийская мисcия выехала из Белграда. 28 июля
Австро -- Венгрия объявила войну Сербии.

И ничего страшного пока не произошло. Очередная бочка пороха рванула на
Балканах.

Но тут уже в ход пошли тяжёлые фигуры Германской и Российской империй. Россия
объявила частичную мобилизацию военных округов, Киевского, Одесского,
Московского и Казанского, нацеленных на Австро -- Венгрию.

Сазонов кинулся к англичанам и французам. Сазонов умолял британцев опубликовать
заявление образца 1911 года, что Англия поддрежит своих союзников. Он надеялся,
что это может остановить хоть на время Берлин. Что из этого получилось?

Французский президент и министр иностранных дел всё ещё неторопливо плыли в
белль Франс и находились вее зоны обслуживания сети. Заменяющий премьера Франции
министр заявил, что Франция \enquote{не собирается драматизировать события}. Министр
иностранных дел Грей порадовал нас мудрой мыслью, что всё это \enquote{давний
конфликт тевтонов и славян}, не имеющий значения для всей Европы. Посол Его
Величества Бьюкенен просто улыбнулся и сообщил, что \enquote{Англия из-за
Сербии воевать не будет}.

Отличное начало для войны, правда? А?! Сторонники верности союзническому долгу,
как вам всё это? Россию швырнули как Ларису Огудалову на \enquote{Ласточке}, а обещали
в Париж. И это ещё и война не началась!

24 июля кайзер Германии сообщил Англии и Франции, что \enquote{ссора Австрии с
Сербией} может быть покончена лишь усилиями этих двух стран, их конфликт должен
быть локализован, ибо всякое вмешательство третьей державы (России) должно
вызвать \enquote{по естественной игре союзов} неисчислимые последствия. Англия и
Франция согласно кивнули германскому гению.

Россия, имея на руках развязного сироту --- Сербию, оказалась чуть ли не
зачинщицей мирового пожара, какой-то заполошной деревенской дурой, забежавшей
в благородное собрание. И такие любезные ранее союзники стали вдруг настолько
джентельменами, что согласились проводить Россию -- матушку на стылую улицу, где
её уже поджидали при свете фонаря дядя -- Вильгельм и дедушка Франц, согласные
уже на частичное удовлетворение своих потребностей, одним востоком, т.е. нами.
Начальник Главного Штаба германской армии Мольтке -- младший произнёс:
\enquote{Мы готовы и теперь, чем скорее, тем лучше для нас}. Создалась ситуация, когда
великий план Шлиффена можно было откорректировать, ликвидировав сначала
восточную угрозу рейху при попустительстве западных российских союзников.

Все колебания Николая Александровича, за которые ему так доставалось и от
современников, и от исследований, все эти жалкие телеграммы кузену Вилли,
метания по кабинету --- не только от известного безволия самодержца. Рушилась
система европейского присутствия России. Повторялся кошмар его прадеда, Николая
Павловича. Россия запуталась в доверии и обманах. Всё что строилось: валютная
система, индустриализация, программа военного обновления --- всё оказывалось
бессмыслецей.

Спас (хотя \enquote{спас} --- это  не очень подходящий термин) ситуацию для России,
уже стоящей перед войной в одиночку в Австрией, тем более опасной, что
готовились мы, прежде всего, к войне с Германией, сам кайзер Вильгельм. Особым
ходом своей мысли (а мысли у кайзера были причудливы как совместные постановки
Линча и фон Триера по сценарию Тарантино), кайзер испугался, что Россия (у
которой разум был, по мнению кайзера), не будет защищать сербов--цареубийц
(подчеркну --- цареубийц, причём рецедивных), что Франция будет удерживать
Россию от войны с Австрией, а Англия останется нейтральной! И что тогда?! Чем
заняться рейху? Для чего было столько стараний? Жертв? И химизации? Вроде как,
всю конструкцию построил, всех гостей на кровавый ужин позвал, а теперь его,
кайзера, отправят крутить шашлыки за сарай?! Не порядок\ldots

28 июля министр Сазонов принял германского посла графа Фридриха фон Пурталеса,
на которой Пурталес сказал: \enquote{Теперь уже поздно}.

Дальше начнётся германская радиоигра с императором Николаем. 28 июля Вильгельм
телеграфирует Николаю, что обещает воздействовать на Вену ради \enquote{нашей
дружбы\ldots} Следующая телеграмма от того же 28 июля:  Вильгельма грубо
советует Николаю не впутывать в свои дела Германию и обращаться непосредственно
к императору Австро -- Венгрии. Все эти телеграммы (воскрешающие метод О.Бендера)
пришли на скромную просьбу нашего самодержца -- мыслителя \enquote{как-то повлиять на
Вену и не дать зайти ей слишком далеко}. Дезориентация Николая шла полным
ходом. Не знаю, ржали ли немецкие телеграфисты, отбивая послания Вильгельма
брату Коле.

Николай хватается за последнюю соломинку: предлагает передать спор между
Австрией и Сербией в Гаагский трибунал или Третейский суд. Хохотала уже вся
Европа, а не только немецкие телеграфисты.

29 июля, пока в Европе ещё хохотали, начальник Главного Штаба российской армии
генерал Янушкевич объяснял полковнику Н. А. Романову азы мобилизационных
процессов. Частичная мобилизация забьет русские железные дороги и сорвёт
возможную общую мобилизацию. От западных союзников многозначительная тишина.
Царь колеблется. Ему не очень хочется воевать в одиночку, но и стать
неподготовленной жертвой агрессии тоже страшно. Никогда ещё перед Николаем
Александровичем не стояло такое сложное управленческое решение, на которое его,
собственно, уполномочил бог и народ. Узнав, что Австрия мобилизуется поностью,
царь утвердил указ о всеобщей мобилизации в России.

За несколько минут до передачи указа по военным округам, царь отменяет своё
решение. Ещё бы, ведь он получил новую телеграмму от кайзера, в которой Вилли
говорит о старой дружбе. Поэтому Никки отменяет всеобщую мобилизацию, но
продолжает мобилизацию войск, направленных против Австрии.

Ночью в Петербург приходит паническая телеграмма русского посла в Берлине
Свербеева: Германия объявила всеобщую мобилизацию, Германия начала общую
мобилизацию. Германия готова напасть на Россию. От союзников России --- тишина.

Утром 30 июля на квартиру к Янушкевичу прибыл военный министр Сухомлинов, по
телефону вызвали Сазонова. Встреча происходила в неофициальной обставке.
Составили план: Сазонов скачет в Петергоф, уговаривать Николая. В случае успеха
уговоров Сазонов звонит Янушкевичу. Янушкевич отбивает телеграммы на главном
телеграфе, потом ломает свой телефон и прячется, пока царь снова не передумал.

Позвонили царю --- царь в Петергофе отдыхает, телефон в петергофском дворце
только в комнате камердинера, под лестницей. Царь спускается в комнату
камердинера, под лестницу. Янушкевич умолил царя принять Сазонова. Царь
назначает министру Сазонову встречу на 15 часов. А дальше уже Россия. Министр
приезжает к абсолютному монарху в минуту наивысшего напряжения сил и опасностей
для империи с 10 минутным опозданием. Говорит около часа. Царь возражает. Потом
взволнованно говорит: \enquote{Сергей Дмитриевич, пойдите и телефонируйте начальнику
Главного штаба (это в комнату, значит, камердинера иди), что я приказываю
провести общую мобилизацию}. Сам решил не ходить больше в камердинерскую,
помазанник.

Всё.

Дальше была уже пустая беготня Пурталеса к царю, телеграммы царя в Вену,
ультиматум Германии России. С опозданием в один час против запланированного
(планировали на 18, случилось в 19 часов) 1 августа 1914 года Германская империя
объявила войну России. Союзники России безмолвствовали.

3 августа немецкие войска вторглись в Бельгию. В это же день Германия объявила
войну Франции. 4 августа Британская империя объявила войну Германии.

Осталось только понять экономическую подоплёку участия в этой каше России?
Уточнить цену вопроса, так сказать.
