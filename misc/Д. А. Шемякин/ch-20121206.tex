30 июля 1914 года (Франция не объявляла даже всеобщей мобилизации, а, напротив
отодвинула свои войска от границы на 10 км, послав нам телеграмму
\enquote{крепить мир}. Берлин предложил Великобритании оставаться нейтральной, если
Германия не будет нападать на Францию, а ударит только по России. Британское
правительство с гневом отбросило германские условия как бесчестие! \enquote{Мы
не могли бы обсуждать даже и сделку за счёт Бельгии}, не то, что за счёт наших
остальных замечательных союзников.

1 августа (в день объявления войны Германией России) министр иностранных дел
Великобритании Эдуард Грей вызвал посла Германии в Лондоне Карла Лихновского и
предложил следующее: Англия сохраняет нейтралитет, если Германия пообещает не
нападать на Францию, а нападёт только на Россиию.

Этот шаг Грея в Берлине встретили в положении полуприседа от радости и неверия
в собственное счастье. Многие ведь до сих пор не верят в то, что у России был
реальный шанс остаться один на один с врагами. Франция ведь тоже ожидала
нападения на себя, она не хотела объявлять войну Германии и Австрии из-за
русско-германских противоречий. А по плану Грея всё выходило чудесно. Россия
уже воююет. Франция только мобилизуется, Германия даёт гарантии ненападаения на
Францию, Англия объявляет о своём торжествнном нейтралитете. Немцы и австрийцы
раскатывают Российскую империю, использую германские силы, забронированные на
западе, перебросив их на восток.

Кайзер был в восторгге! Франц-Иосиф, проинформированный из Берлина об
иницитативе Лондона, тоже взбодрился до крайности. Петербург несколько
растерянно молчал. Франция в лице своего посла Камбонга ограничилась указанием
на то, что в случае победы над Россией, усилившаяся Германия всё равно нападёт
на Францию и тогда Лондон будет ждать (будет ждать, да) возмездие от победившей
Германии или победившей Франции. Что там за возмездие посол не уточнял. Понятно,
что никакого возмездия от двух обескровленных государств Британии не грозило бы
минимум лет десять.

Повторю  --- Франция не собиралась сама начинать войну с Германией из-за
русско -- автрийско -- германских противоречий. Франции нужно было нападение
Германии или на неё саму, или на Бельгию. Англия предложила решить вопрос за
счёт перенаправления сил рейхсвера на восток, сохранения нейтралитета Бельгии
и, соответственно, своего полного невмешательства.

\vspace{1ex}\noindent
Так всё удачно!
\vspace{1ex}

Как всегда, свою чудесную партию на ударных сыграли не\-мец\-кие военные. Они
изловили радующегося кайзера и объяснили, тыча пальцами, что по плану (по
немецкому, самому лучшему плану на свете) первый удар должен быть нанесён по
Франции с обязательным вторжением в нейтральную Бельгию. Иначе, говорили
генералы, это что за война такая получается? Не на два фронта, без нарушения
международных норм, неинтресная война будет! Плюс нам план придётся
переделывать. Мольтке-младший даже пригрозил кайзеру своей отставкой, считая
что день перемены плана войны будет гибельным для Германии. Немецкие
генштабисты гурьбой носились за германскими полтиками (некоторые политики даже
думали прятаться от генералов), тряся папками и выкладками. План-то под угрозой!
Такие сцены мы можем легко представить, посмотрев советские фильмы про сдачу
цементного завода под Новый год с обязательным актёром Ульяновым в главной роли.
\enquote{Ты мне рыцарский крест на стол положишь! Если хоть на день план
сдвинешь! Понял, Федосеев, твою мать?!}

С тяжёлым сердцем Берлин ответил отказом на предложение Лондона. Немцы,
потупясь, сказали, что всё же будут нападать на Францию, что нет сил удержать
это естественное желание, лежит, такая розовая, такая вся в истоме, на берегу
Сены, как не напасть зольдату на такое великолепие?! И на Бельгию нападут!
Потому, что по плану!

Грей получил отказ из Берлина, собрал всех членов кабинета на уик-енд и сообщил,
что воевать, наверное, прийдётся. Пригласили на встречу членов британского
парламента, готовых голосовать за войну (их не было большинство). Члены
парламента готовы были начать душить Грея прямо в кабинете, но тут на открытии
сессиии парламента в зал вошёл неприметный человек и сообщил, что германские
войска только что вторглись в \enquote{маленькую, бедную, обиженную Бельгию}.
3 августа британский кабинет решил объявить войну Германии, используя мотив
(подчеркну) защиты Бельгии (не выполнение союзнических обязательств перед
Россией, а защиту крошки --- Бельгии). Три министра британского правительства
немедленно вышли в отставку в знак протеста против такой несправедливой войны.
3го августа в Берлин послали ультиматум британской стороны. Что было в
британском ультиматуме? Требование прекращения войны с Россией, требование
гарантий для Франции? Нет. В ультиматуме было требование соблюдать нейтралитет
Бельгии. И всё. На этот сентиментальный ультиматум Германия даже не ответила,
потому что у немцев был всемогущий и жестокий план! И только 4 августа Грей
отправил в германское посольство письмо о том, что Великобритания находится в
состоянии войны с Германией. Понимаете, письмо он послал, ладно под дверь не
подсунул. Никаких нот, никаких волнующихся и плачущих послов (как было у нас
между Сазоновым и Пурталесом). Письмо направил, расстроенный был.

Вот с такими союзниками мы поехали бить городских.
