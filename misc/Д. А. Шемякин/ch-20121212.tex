Как мы все прекрасно знаем, одной из самых удачных финансовых операций
Российского государства была отмена крепостного права в 1861 году.

Не затратив ни рубля, российское правительство получило 700 миллионов рублей
чистого финансового выигрыша. Странно, что люди, обожающие свободу во всех её
проявлениях, не используют лозунг о том, что освобождение - это государственная
выгода. Странно, что люди, свободу во всех её проявлениях не обожающие, не
используют лозунг, что свобода --- это затратно и невыгодно.

\noindent Вооружимся, наконец, счётами.

19 февраля 1861 года началось освобождение частновладельческих крестьян. Великая
реформа. В ходе этой реформы правительство выплатило помещикам стоимость
переданной освобождённым крестьянам земли. Оценили эту переданную крестьянам
землю в 1 миллиард 218 миллионов рублей. Потом чиновники (которые сами в
значительной части не из помещиков происходили) вычли из этой, в целом,
огроменной суммы задолженность помещиков государственным кредитным учреждениям.
Так заминусовали 316 миллионов рублей. И помещики должны были получить
\enquote{на руки} за землю, переданную крестьянам --- 902 миллиона. Тоже неплохо
и несколько даже радостно. Но государственные люди подумали и решили, что
помещики столько \enquote{живых денег} не осилят --- это вредно столько
наличности на руках держать человеку непривычному. Поэтому урезанная до 902
миллионов рублей сумма выдавалась помещикам не деньгами. А специальными ценными
бумагами (5\%-ми выкупными свидетельствами). И естественно, что бумаги,
выброшенные правительством в оборот, а помещиками --- на рынок, оценивались
первоначально ниже номинальной стоимости, примерно процентов на 20-ть ниже.

Внук, сын помещиков, сам помещик, брат первого в России изготовителя бормотухи
\enquote{ярославской фабрикации}, которой спаивают несчастного Карандышева в
\enquote{Бесприданице}, великий  поэт Н. А. Некрасов, когда писал вот это:

\begin{verse}
Повалась цепь великая,\\
Повалась --- расскочилася:\\
Одним концом по барину,\\
Другим по мужику!\\
\end{verse}

\noindent он ведь писал кровью сердца своего.


И его можно понять. Как ударила реформа по крестьянам --- это, в принципе,
понятно: русский крестьянин за вековое своё состояние привык к неволе, свободу
ценил не очнь высоко, так как не очень понимал, что это такое, а вот землю,
единственный источник существования для большинства селян, ценил очень. И деньги
ценил. А вот барин, он что потерял? Представить себе это тоже просто: приходят к
москвичу люди из мэрии и говорят, что квартиры на набережных, доставшиеся от
дедушек с бабушками, и  которые москвич сдаёт иногородним менеджерам, теперь
будут частью уже и не его, что деньги он за это получит, но потом, наверное.
А пока москвич получит на руки бумагу очень ценную, которую, если сможет,
продаст по рыночной стоимости, которая, правда, ниже номинала, но не беда. И
потом 5 процентов личнозависимых крестьян составляли обслуживающий персонал
помещиков, это же целый сектор услуг самого разного свойства: от дизайна,
транспорта и развлечений до финансового личного менеджмента. И теперь за весь
спектр услуг надо платить, а то останешься только с Фирсом образца финала
\enquote{Вишнёвого сада}, с таким дизайнером и шофёром не зашикуешь.

Получив на руки за часть своей земли ценные бумаги вместо денег, помещики
пережили, как пишет князь Мещерский, историческую по интересу минуту.
Большинство кинулось продавать свои выкупные ценные бумаги, теряя при продаже
от 18 до 24 процентов номинальной стоимости. Тут я обращаю ваше внимание, что
цену земли, из которой рассчитывалась номинальная стоимость выкупного
свидетельства, назначал не сам хозяин, а справедливое государство, исходя из
\enquote{статистики}. Видя такое дело, большинство поместного дворянства решило
не дожидаться ничего хорошего, сделать экономическую ставку на земельный фонд,
оставшийся в их распоряжении (к концу 19 века помещики в России владели
территорией размером с Францию), а ценные бумаги за землю, преданную
крестьянству \enquote{вложить с умом}. В русской традиции это может означать что
угодно. Я знаю огромное количество людей, которые в 90-е годы с умом вкладывали
деньги в видеомагнитофон и шубу для жены, называли это вложением денег вполне
серьёзно. Как вспоминает тот же князь Мещерский \enquote{была блестящая эта
эпоха выкупных свидетельств. В каждую семью, в каждый дом сваливались с неба
крупные тысячные суммы\ldots и вот эти-то выкупные продавались и превращались в
капиталы, на которые одни бросились в заграничные поездки, а другие стали жить
очень роскошно в Петербурге и в Москве\ldots Только меньшая часть тогда
владельцев выкупных, ввиду низкой их цены, решилась выжидать повышения цен;
большая же часть с изумительной лёгкостью бросилась их реализовывать\ldots}

Правительство, конечно, некоторое время, например, раздувая известия о
крестьянских беспорядках, активно играло на понижение курса \enquote{выкупных
5\%}.

Но не забывало правительство и об освобождённом крестьянстве. Общая сумма
выкупных платежей, полученная правительством с бывших помещичьих крестьян с
1861 года по 1906 год оставила 1 миллиард 600 миллионов рублей. Как я уже писал
вся эта чудесная сумма попала правительству в любящие руки при условии, что эти
любящие руки не вложили копейки в проведение освобождения частновладельческого
крестьянства. А ведь помимо выкупных денег с крестьян (с крестьян решили брать
деньгами живыми, решили, что, если крестьяне начнут выпускать свои ценные
бумаги, то будет как-то не интересно), государство сохранило черезвычайно
серьёзное налогообложение села. В совокупности, с деревни правительство получало
налоги, в три раза превосходящие в совокупности налоги со всех отраслей
промышленности. Только в 1901 году сумма прямых налогов с надельных крестьянских
земель составила 284 миллиона да и 300 тыс. рублей в довесок.

Какие там евреи -- кровососы?! Правительство и без них отлично справлялось с
финансовым доением мужичка. Да и дворянское вымя правительство щупало тоже
весьма дерзко.

Конечно, наделы у крестьян после освобождения оказались маленькие. На таких
наделах товарное производство не наладишь. Поэтому крестьяне были вынуждены
арендовать землю, оставшуюся в собственности у помещиков. Уже после всех этих
столыпинских нововведений, в 1914 году 80\% всех посевов сельскохозяйственных
культур принадлежали крестьнам, но посевы эти в подавляющей значительностью
степени происходили на помещичьей земле, арендованной крестьянами у владельцев
на условиях, которые министерство финансов империи жалостливо называло
\enquote{крайне тяжёлыми}. Такой вот аграрный вопрос встал перед крестьянством.
Что ж нам делать, православные? Куда податься?

Очень многие любители \enquote{жизни за царя} указывают мне на рост валового
сбора хлеба и рост хлебного экспорта. Вот, мол, смотри, как деревня при царе --
батюшке заиграто жила! Аргументация у таких людей серьёзная. К середине 90-х
годов 19 века сбор хлеба вырос в сравнении с 70-ми годами 19 века на 44\%
(2,6 миллиардов пудов в сравнении с 1,8 миллиарда пудов). С начала 1880 до
конца 1890-х экспорт зерновых повысился в два с половиной раза и перешагнул
планку в 500 миллионов пудов. К 1913 г. валовый сбор зерна вырос до 5 миллиардов
пудов, а экспорт зерна вылился в 650 миллионов пудов. Рекорд Россия поставила в
1910 году, вывезя 847 миллионов пудов хлеба за границу.

\noindent Что я могу ответить на это? Немногое.

\begin{enumerate}
\item Рост показателей обспечивался в значительной степени перемещением
(колонизацией) зернового производства из России центральной на перефирию, где
русской деревни в её классическом понимании не было. Крестьяне устремлялись
сначала почти хаотически, потом уже почти централизовано на свободные земли,
связанные с окружающим миром, с Россией центральной, с обычаями, укладом и
ограничениями, торговыми ветками специальных железных дорог как частных, так и
всё более и более казённых.
\item Естевенно прогресс в сельском хозяйстве был, чего тут скрывать. Но
прогресс этот надо сравнивать. К началу 20 века среднегодовая урожайность ржи в
крестьянских хозяйствах составляла 53 пуда с десятины, в Германии --- 104 пуда с
десятины. Урожайность пшеницы --- 59 пудов с десятины для России и 126 пудов для
Германии. Бельгия смотрела на это негласное соревнование, сидя на среднегодовой
урожайности ржи в 142 пуда с десятины, а урожайность пшеницы в Бельгии была 153
пуда с десятины.
\item Что же происходило с хлебушком под благодетельным имперским солнышком в
России? Как его потребляли русские? Немного они его потребляли. В начала 20 века
Россия могла себе позволить в год 22,4 пуда хлеба на душу населения, Германия
--- 24 пуда, Дания --- 50 пудов. И мы должны ещё учесть, что каждый второй ---
третий год в России был неурожайным и норма потребления падала до 16 пудов на
душу. По экономическим показателям, по медицинским показателям (учитывая
крестьянский рацион) это был критический уровень. Чтобы не было тайны ---
тароватая, обильная Россия, румянощёкая красавица с косой толщиной в руку,
славящая в песнях своих великую Российскую империю находилась на одном из
последних мест в Европе по уровню потребления простого хлеба. Мы гордо опережали
лишь Румынию и ещё несколько стран предрумынского состояния. При полусытости
своей, при своём полуголоде Россия экспортировала 11,6\% хлебного сбора в 80-е
годы 19 века, США при высокой норме потребления населением вывозила 8\%.
Посланник США в России Смит в 1891 году писал, что Россия искусственно
стимулирует у себя экспорт, уступая многим странам в производстве зерна. В этом
Смит видел секрет пребывания России в состоянии \enquote{почти хронического
голода}.
\item Радетели империи! Россия вывозила свой хлеб не из-за благоденствия своего,
не из-за высокого уровня сельскохозяйственного производства, а из-за
экономической слабости крестьянских хозяйств. Крестьяне были вынуждены продавать
свой урожай без стратегического остатка, без понимания коньюктуры рынка,
продавать вслепую, второпях, неграмотно армии \enquote{хлебных жуков},
нацеленных на скорейшее получение прибыли без долговременных вложений в сельское
хозяйство. Этими \enquote{хлебными жуками} были не только какие-то там аферисты,
не только хлебные миллионеры, опухшие от безнаказанности, но и, внимание, такие
учреждения, которым вобще не очень свойственно заниматься подобными вещами.
Основными игроками на рынке выкручивания слабых крестьянских рук были банки.
Прямая деятельность иностраннных банков в России по российскому законодательству
была запрещена. Поэтому на русской деревне грелись свои же. Русско -- Азиатский,
Азовско -- Донской, Петербургский международный, Русский для внешней торговли
банк, Русский Торгово -- Промышленный банк. Не оставался в стороне и сам
Государственный банк Российской империи, который выступал ещё как и самый
сильный комммерческий имперский банк. Когда банки занимаются хлебной
спекуляцией, банки не заинтересованы в создании крупных капиталистических
аграрных хозяйств, для спекуляций банкам нужна была Россия в аграрном смысле
полуоформившаяся, которой можно платить не столько, сколько можно или нужно, а
сколько хочется. Как на дискотеке в ночном клубе \enquote{Огонёк}. В Европе
банки хлебным бизнесом не промышляли. Исключение, выгодно оттеняющее
деятельность российского банковского капитала --- это Колониальный Банк
Великобритании, деятельность которого в колониях была аналогичной деятельности
российских банков в России. И конечно, активно играли на рынке скупки
продовльствия банки Индии и некоторых латиноамериканских стран.

А поиграть банкам России было с чем. С одной стороны, нищая, вынужденная
продавать хоть как-то своё зерно деревня, с другой стороны суммарный основной
капитал коммерческих банков в России увеличивался с 236 млн. рублей до 561 млн.
рублей только за три года (с 1910 по 1913 гг). Естественно, что операции на
хлебном рынке были не основным занятием банков, но раз есть такая возможность,
что ж не вложиться-то, не посбивать цены на внутреннем рынке, расталкивая
прочих?

А крестьянин что? Он свободный! Хочет прыгает на своём клочке земли, хочет берёт
землю в аренду у барина из Парижа (платит деньгами или \enquote{отработкой}),
хочет бредёт в степи, чтобы на чернозёме начинать жизнь заново, хочет в город
идёт, хочет граммофон у трактира слушает. Вяльцеву или даже Шаляпина. Если
повезёт и будет урожай, то крестьянин никуда его продавать поехать не сможет,
т.е. поедет, конечно, но не на биржу же зерновую, не фьючерсами бомбиться, а
приедет он в ближайший уездный город и продаст там ловкому человеку -- агенту в
блестящей жилетке весь свой урожай, оставив на пропитание чуть-чуть. Крестьянину
ведь, как это не удивительно, деньги нужны. И не только на водку. Ему, например,
нужны орудия сельхозпроизводства.
\item Уровень механизации сельского хозяйства России (соотношение машин,
механизмов, с одной стороны и \enquote{живой} рабсилы --- крестьянин плюс лошадь
--- с другой) накануне Первой мировой составлял 24\%. В Англии механизация
аграрного сектора составляла 152\%. В Германии --- 189\%. В США --- 420\%. И при
этом вот чуде живого ковыряния живой земли, по данным самого Петра Аркадьевича
Столыпина, расходы государства на сельхозпомощь населению (покупка парового
сепаратора или веялок - это же для деревни было как в космос слетать, без помощи
никак) достигали: в Норвегии и Венгрии --- 2 рубля на десятину посевной площади,
в Пруссии --- 1 руб. 33 коп., в Бельгии --- 1 рубль, в Европейской России,
стоящей в аграрном отношении где-то очень там, глубоко внизу, православное
государство вкладывало в модернизацию и поддержку русского крестьянского
хозяйства - 9 (девять прописью) копеек на тощую десятину.
\end{enumerate}

Так процветала имерия Романовых, имея в фундаменте 80 процентов недоедающего
недоброго населения.

Но деревню разорили, конечно, большевики с Лениным -- Сталиным.

А ещё это мой ответ на постоянно слышащиеся полужалобы -- полухвальбы про
какую-то дикую \enquote{русскую лень}. Русская лень, о которой столько
написано --- это обычное отчаяние, явственное понимание крестьянским крепким
умом отсутствия перспектив, в головах ведь не только разруха, там ещё и
пауперизация отлично себя может чувстсвовать. Вот возьмите кредитов на жильё,
снимая квартирку для семьи, подсчитайте свои возможности и перспективы, череду
дальнейших лет за копейки, а я мимо проеду на своём \enquote{БМВ} и расскажу
вам, какие вы ленивые да убогие, раз на пятую работу не устроились, в съёмной
хате своими силами финский паркет не укладываете.
