Русско -- германские экономические противоречия накануне Первой Мировой.

Трудно писать об этом, но русско -- германские экономические противоречия
накануне первой мировой войны были, как бы сказать, странноватыми.

Понятно, что Германия с конца 19 века активно вторгается в международную
торговлю, обеспечив себе практически неисчерапемые железно -- рудные и уголньные
запасы Эльзаса и Лотарингии, стартовый капитал в пять миллиардов франков
контрибуции и высочайший уровень культуры производства. Как и полагается стране
\enquote{форсированного} развития Германия сделала ставку на создание и
развитие базовых отраслей тяжёлой (очасти это значит и военной) промышленности.
Высокие технологии, энерговооруженность, металлургия, машиностроение,
железнодорожное строительство, угледобыча, химия и т.п. выводят страну на
передовые позиции.

К 1910 году удельный вес промышленности в Германии составил 37,9 процентов
национального хозяйственного потенциала, 35 процентов составлял удельный вес
сельского хозяйства интенсивного типа. С 1870 по 1913 гг. суммарный объём
промышленной продукции Германии вырос на 471 (четыреста семьдесят один) процент.
В Англии этот показатель составил 27 процентов, во Фарнции - 203 процента.

Вершина, достигнутая кайзеровской Германией --- это первое место в Европе и
второе место в мире по объёму промышленного производства. Германская империя
давала мировой объём промышленного производства в размере 16,5 процентов.

Сельское хозяйство рейха обеспечило рост урожайности ржи с 1870 по 1913 гг. в 2
с половиной раза. Только в Восточной Пруссии посевные площади только зерновых
расширились на 36 процентов.

Кенигсберг вытеснил с рынка первозок на Балтике западные порты России, в первую
очередь, Либаву.

Вывоз германского капитала в страны, в которых был заинтресован германский
капитал увеличился за 13 лет (с 1900 по 1913 гг.) с 12,5 до 44 млрд. марок.

Есть мнение, что Росия бешенно зависела от французского и бельгийского
капиталов, от займов и т.п. Не отрицая ничего, отмечаю, что 20 процентов
инвестиций в русское народное хозяйство (т.е. капиталы, вложенные в производящие
отрасли России) были германские.

К началу 20 века Германия имела 12,6 процентов мирового товарооборота. На долю
российской империи приходилось 4,2 процента мирового экспорта и 3,5 мирового
импорта.

Рынок России был одним из самых важнейших потребителем германского промышленного
оборудования. Россия переживала тогда индустриализацию, германская
промышленность обеспечивала ускорение этой индустриализации.

Вот на этом некая идиллия заканчивается. Дальше начинается хардкор как он есть.
